\documentclass[11pt]{article}

\usepackage{texyousei}
\usepackage{tabularx}

\lhead{}
\chead{The Tangent-Point Energy}
\rhead{}

\usepackage{comment}
\excludecomment{oops}

\setlength{\leftmargini}{0.3in}

\DeclarePairedDelimiter{\inner}{\langle}{\rangle}
\newcommand{\ddx}{\frac{d}{dx}}
\newcommand{\ddy}{\frac{d}{dy}}
\newcommand{\ddn}{\frac{d}{dn}}
\newcommand{\ddm}{\frac{d}{dm}}
\newcommand{\sgn}[1]{\ \textrm{sgn}\left(#1\right)}
\newcommand{\calDf}{\mathcal{D}_f}

\begin{document}

\maketitle{The Tangent-Point Energy on Surfaces}{}

\thispagestyle{empty}

\section{Overview}

On surfaces, the tangent-point energy has a slightly simpler expression, due to the ambient dimension of 1 (versus 2 for curves). The kernel can be written as:
\begin{equation}
K_f(x, y) = \frac{\inner{N(x), f(x) - f(y)}^{\alpha}}{\norm{f(x) - f(y)}^{\beta}}
\end{equation}
Because there is a unique normal direction, it suffices to use the inner product with the normal, instead of the cross product (as we used for curves).

\subsection{Differential}

Still to be written down...

\end{document}

