\documentclass[11pt]{article}

\usepackage{texyousei}
\usepackage{tabularx}

\lhead{}
\chead{The Tangent-Point Energy}
\rhead{}

\usepackage{comment}
\excludecomment{oops}

\setlength{\leftmargini}{0.3in}

\DeclarePairedDelimiter{\inner}{\langle}{\rangle}
\newcommand{\ddx}{\frac{d}{dx}}
\newcommand{\ddy}{\frac{d}{dy}}
\newcommand{\ddn}{\frac{d}{dn}}
\newcommand{\ddm}{\frac{d}{dm}}
\newcommand{\sgn}[1]{\ \textrm{sgn}\left(#1\right)}
\newcommand{\calDf}{\mathcal{D}_f}

\begin{document}

\maketitle{The Tangent-Point Energy on Surfaces}{}

\thispagestyle{empty}

\section{Overview}

On surfaces, the tangent-point energy has a slightly simpler expression, due to the ambient dimension of 1 (versus 2 for curves). The kernel can be written as:
\begin{equation}
K_f(S, T) = \frac{\abs{\inner{N_f(S), X_f(S) - X_f(T)}}^{\alpha}}{\norm{X_f(S) - X_f(T)}^{\beta}}
\end{equation}
Because there is a unique normal direction, it suffices to use the inner product with the normal, instead of the cross product (as we used for curves). Here we use $X_f(S)$ and $X_f(T)$ to denote the barycenters of triangles $S$ and $T$ under the embedding $f$.

\subsection{Differential}

Let $A$ be the numerator of $K_f(S, T)$, and let $B$ be the denominator. Then we just need $\ddx A$ and $\ddx B$ to be able to assemble the full differential $\ddx K_f(S, T)$ using the product rule, where $x$ is a vertex position.

First we differentiate $A$:
\begin{align*}
\ddx A &= \ddx \abs{\inner{N_f(S), X_f(S) - X_f(T)}}^{\alpha}\\
&= \left(\alpha \abs{\inner{N_f(S), X_f(S) - X_f(T)}}^{\alpha - 1}\right) \left(\ddx \abs{\inner{N_f(S), X_f(S) - X_f(T)}}\right)\\
&= (\cdots) \sgn{\inner{N_f(S), X_f(S) - X_f(T)}} \ddx \inner{N_f(S), X_f(S) - X_f(T)}\\
&= (\cdots) \sgn{\cdots} \left(\ddx N_f(S) \left(X_f(S) - X_f(T)\right) + \ddx \left(X_f(S) - X_f(T)\right) N_f(S)\right)
\end{align*}
Both of the remaining inner derivatives have simple expressions. For the normal, we have
$$\ddx N_f(S) = \left\{
\begin{array}{lr}
\frac{1}{2 A_f(S)} (e_{x,S} \times N_f(S)) N_f(S)^{\top} & x \in S \\
0 & x \not\in S
\end{array}\right.$$
where $e_{x,S}$ is the vector along the opposite edge from vertex $x$ in triangle $S$. And for the barycenter, we just have a Jacobian with $1/3$ along the diagonal if $x \in S$, and 0 otherwise.

Next, we differentiate $B$:
\begin{align*}
\ddx B &= \ddx \norm{X_f(S) - X_f(T)}^{\beta}\\
&= \beta \norm{X_f(S) - X_f(T)}^{\beta - 1} \ddx \norm{X_f(S) - X_f(T)}\\
&= \beta \norm{X_f(S) - X_f(T)}^{\beta - 1} \frac{X_f(S) - X_f(T)}{\norm{X_f(S) - X_f(T)}} \ddx \left(X_f(S) - X_f(T)\right)
\end{align*}

Finally, we just write down the quotient rule:
$$\ddx K_f(S, T) = \ddx \frac{A}{B} = \frac{(\ddx A) B - (\ddx B) A}{B^2}$$


\end{document}

